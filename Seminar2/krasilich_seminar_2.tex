\documentclass{article}[14pt]
\usepackage[utf8]{inputenc}
\usepackage[T2A]{fontenc}
\usepackage[russian]{babel}
\usepackage{amsmath}
\usepackage{amsfonts}
\usepackage{hyphenat}
\usepackage{tikz-cd}
\usetikzlibrary{babel}
\usepackage{letltxmacro}
\usepackage{amsthm}
\usepackage{amssymb}
\usepackage{tikz-cd}
\usepackage{proof}
\hyphenation{ма-те-ма-ти-ка вос-ста-нав-ли-вать}

\newtheorem{theorem}{Theorem}
\newtheorem{proposition}{Proposition}
\newtheorem{definition}{Definition}
\newtheorem{corollary}{Corollary}

\title{Math Methods of Science \\
Seminar 2}
\author{Gleb Krasilich}
\date{October 2021}

\begin{document}

\maketitle

\section{Problem 1}

In the Problem 3 of home task 1 I have already shown that direct sum of vector bundles is a smooth manifold and a vector bundle.
So the only thing we left to check is that map $f: (x, v_1, v_2) \mapsto (x, v_1 + v_2)$ is a morphism of vector bundles.\

Indeed, the diagram
\[
\begin{tikzcd}
    E \oplus E \arrow[rr, "f"] \arrow[rd, "\tilde{\pi}"] & &  \arrow[ld, "\pi"] E \\
    & X & \\
\end{tikzcd}
\]
by definition of $f$.
$f$ is acting linearly fiber wise: if $(v_1, v_2)$ and $(w_1, w_2)$ are vectors in 
$\mathbb R^r \oplus \mathbb R^r$, then $f((x, v_1 + w_1, v_2 + w_2)) = (x, v_1 + w_1 + v_2 + w_2) = (x, (v_1 + v_2) + (w_1 + w_2))$,
and $f((x, \alpha v_1, \alpha v_2)) = (x, \alpha v_1 + \alpha v_2) = (x, \alpha (v_1 + v_2))$ for any $\alpha \in \mathbb R$.

Finally, $f$ is surjective on each fiber: for any vector $v$ and point $x \in X$, $(x, v) = f((x, v, 0))$.
Therefore $f$ has constant rank equal to the rank of $E$ at each fiber.
So $f$ is a morphism of vector bundles. Q.E.D.

\section{Problem 2}

Let $g: \mathbb R^n \rightarrow T_x X$ be a linear function which maps $(v_1, \dots, v_n)$ to $v_1 \frac{\partial}{\partial x_1} + \dots + v_n \frac{\partial}{\partial x_n}$.
First, let us show that $g$ is injective. 


\end{document}