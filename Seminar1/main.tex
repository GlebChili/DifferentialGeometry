\documentclass{article}[14pt]
\usepackage[utf8]{inputenc}
\usepackage[T2A]{fontenc}
\usepackage[russian]{babel}
\usepackage{amsmath}
\usepackage{amsfonts}
\usepackage{hyphenat}
\usepackage{tikz-cd}
\usepackage{letltxmacro}
\usepackage{amsthm}
\usepackage{amssymb}
\usepackage{tikz-cd}
\usepackage{proof}
\hyphenation{ма-те-ма-ти-ка вос-ста-нав-ли-вать}

\newtheorem{theorem}{Theorem}
\newtheorem{proposition}{Proposition}
\newtheorem{definition}{Definition}
\newtheorem{corollary}{Corollary}

\title{Math Methods of Science \\
Seminar 1}
\author{Gleb Krasilich}
\date{October 2021}

\begin{document}

\maketitle

\section{Problem 1}

Let us show that function $s_1 + s_2$ is smooth and well-defined.
Let $x \in X$ be an arbitrary point and $U_x$ be a trivializing neighborhood of $x$.
Without loss of generality we can assume, 
that $U_x$ is diffeomorphic to some open subset $ U \subset \mathbb R^n$ 
(by taking an intersection with a chart containing $x$, if necessary).
Hence $\pi^{-1} (U)$ is diffeomorphic to $U \times \mathbb R^r$.

Now we can define $s_1 + s_2$ in local coordinates as follows: 
for any $z \in U$ if $s_1(z) = (z, v_1) \in U \times \mathbb R^r$
and $s_2 (z) = (z, v_2) \in U \times \mathbb R^r$, 
then $(s_1 + s_2)(z) = (z, v_1 + v_2)$.
Since $s_1$ and $s_2$ are smooth by definition,
$s_1 + s_2$ is, obviously, also smooth in our local coordinates.

So the only thing we left to check is that $s_1 + s_2$ is well-defined globally.
Let $U_x$ and $U_y$ be two trivializing neighborhoods on $X$ with non-empty intersection
$U_{xy} = U_x \cap U_y$. If $g$ is a transition function on $U_{xy}$ and 
$p \in U_{xy}$, then $\Phi_x \circ s_i |_{U_x}(p) = (p, v_i)$ and
$\Phi_y \circ s_i |_{U_y}(p) = (p, u_i)$ where $u_i = g(p) v_i$ by definition of vector bundle.
On other hand, 
$\Phi_y \circ (s_1 + s_2) |_{U_y}(p) = (p, g(p)(v_1 + v_2)) = (p, g(p)v_1 + g(p)v_2) = (p, u_1 + u_2)$.
This indeed shows that map $s_1 + s_2$ is well-defined globally on $X$.

It worth nothing to see that $\pi \circ (s_1 + s_2) = id$. Thus $s_1 + s_2$ is a smooth section.

Similarly we can show that $\alpha s$ is a smooth section for any $\alpha \in \mathbb R$
and $s \in \Gamma(E)$. First, we define $\alpha s$ locally as 
$\Phi_x \circ (\alpha s) |_{U_x} (x) = (x, \alpha v)$ for each $x \in X$ and its trivializing neighborhood.
Then we show that $\alpha s$ is well-defined globally using linearity of transition functions at each point.

\section{Problem 2}

First of all, let us note that $C^\infty (X)$ is indeed a commutative ring with  \\ identity:
addition and multiplication are defined pointwise with constant 0 and 1 functions being correspondently
zero and identity elements. From the problem 1 it also follows that $\Gamma (E)$ is additive 
abelian group.

Given a smooth function $a \in C^\infty (X)$ and a smooth section $s \in \Gamma (E)$,
we define product $as$ locally as $\Phi_x \circ as |_{U_x} (x) = (x, a(x) v_1)$
in terms of points of $X$ and their trivializing neighborhoods.
As we noted in the previous problem, this also shows that $as$ are smooth in local coordinates.
And again as in a problem 1 we can show that this is a well-defined global construction using pointwise
linearity of transition functions: for any $x$, $y$, and $p$ in $X$, if $p \in U_{xy}$,
then $\Phi_{U_y} \circ as |_{U_y} (p) = (p, g(p)(a(p)v_1)) = (p, a(p)g(p)v_1)$.

From the local definition above it is also immediately clear that axioms of module are
satisfied for any chart. But since transition functions at each point are linear, module
operations are well defined globally. For example, we can check that $(\alpha + \beta)s = \alpha s + \beta s$
for each $\alpha, \beta \in C^\infty (X)$ and $s \in \Gamma (E)$ locally compatible on 
each trivializing neighborhoods intersections: 
$\Phi_y \circ ((\alpha + \beta) s) |_{U_{xy}} (p) = (p, g(p)((\alpha(p) + \beta(p) v)) = (p, \alpha(p)g(p)v + \beta(p)g(p)v)$.

\section{Problem 3}

Let $\pi_1: E_1 \rightarrow X$ be vector bundle of rank $r$ and 
$\pi_2: E_2 \rightarrow X$ be a vector bundle of rank $l$.
First, let us define a direct sum $E = E_1 \oplus E_2$.
Let us define $E$ as a disjoint union $\bigcup_{x \in X} \{ x \} \times \mathbb R^r \oplus \mathbb R^l$.
Then for each $x \in X$, if $U_x'$ is a trivializing neighborhood of $x$ for $E_1$
and $U_x''$ is a corresponding trivializing neighborhood for $E_2$,
then we define $U_x$ as an intersection $U_x' \cap U_x''$.

Now let $\pi$ be a surjection $\pi: E \rightarrow X, (x, v) \mapsto x$.
Then $V_x = \pi^{-1} (U_x)$ bears a unique topology, such that $\pi |_{V_x}: V_x \rightarrow U_x$
is a homeomorphism. So we can equip $E$ with a final topology induced by a family
of trivial inclusions $i_x: V_x \rightarrow E$ indexed by points of $X$. Thus $E$ has
a structure of a topological vector bundle, with manifold structure given by an atlas
induced from sets $V_x$.

To complete the definition of smooth vector bundle structure on $E$ we need to define
transition functions. If $g'$ and $g''$ are transition functions for $E_1$ and $E_2$ 

\end{document}
